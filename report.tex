\documentclass[
    iai, % Saisir le nom de l'institut rattaché
    eai  % Saisir le nom de l'orientation
]{heig-tb}

\usepackage[european,americaninductors]{circuitikz}

\makenomenclature
\makeglossaries
\makeindex

\addbibresource{bibliography.bib}



\nomenclature[A, 02]{\(c\)}{\href{https://physics.nist.gov/cgi-bin/cuu/Value?c}
  {Vitesse de la lumière dans le vide}
  \nomunit{\qty{299792458}{\meter\per\second}}}

\nomenclature[A, 03]{\(h\)}{\href{https://physics.nist.gov/cgi-bin/cuu/Value?h}
  {Constante de Planck}
  \nomunit{\qty[group-digits=none]{6.62607015e-34}{\joule\per\hertz}}}

\nomenclature[A, 01]{\(G\)}{\href{https://physics.nist.gov/cgi-bin/cuu/Value?bg}
  {Constante de gravitation universelle}
  \nomunit{\qty[group-digits=none]{6.67430e-11}{\meter\cubed\per\kilogram\per\second\squared}}}

\nomenclature[B, 03]{\(\mathbb{R}\)}{Nombres réels}
\nomenclature[B, 02]{\(\mathbb{C}\)}{Nombres complexes}
\nomenclature[B, 01]{\(\mathbb{H}\)}{Quaternions}

\nomenclature[C]{\(V\)}{Volume constant}
\nomenclature[C]{\(\rho\)}{Indice de frottement sec}

% Notez vos acronymes ici, par exemple:

\newacronym{gcd}{GCD}{Plus grand diviseur commun}
\newacronym{lcm}{LCM}{Plus petit multiple commun}

\newglossaryentry{heig-vd}{
    name=HEIG-VD,
    description={Haute École d'Ingénierie et de Gestion du canton de Vaud}
}
\newglossaryentry{hes-so}{
    name=HES-SO,
    description={Haute École Supérieure de Suisse Occidentale}
}
\newglossaryentry{latex}{
    name=latex,
    description={Un langage et un système de composition de documents}
}
\newglossaryentry{maths}{
    name=mathematics,
    description={Les mathematiques sont ce que les mathématiciens fonts}
}
% Auteur du document (étudiant-e) en projet de Bachelor
\author{Yves Chevallier}

IIDE, REDS, IAI, IGT, MEI, IICT, INSIT, IESE, COMATEC

% Acronyme de l'institut
\institute{IAI}
\endyear{2021}
\department{TIN}
\faculty{Génie électrique}
\orientation{Électronique embarquée et mécatronique}
\thesis{7212}

% Choisir l'option idoine (utilisé pour l'accord féminin)
\genre{male}
%\genre{female}

% Choisir l'option idoine
\field{Ingénierie}
%\field{Economie d'entreprise}

\title{Titre de projet de Bachelor}
\subtitle{Travail de Bachelor}

% Nom du professeur responsable
\teacher {Prof. Y. Chevallier (HEIG-VD)}


%}[bgcolor=bg]
%\definecolor{bg}{rgb}{0.95,0.95,0.95}
% \BeforeBeginEnvironment{minted}{\begin{mdframed}}
% \AfterEndEnvironment{minted}{\end{mdframed}}
\surroundwithmdframed{minted}

%% Début du document
\begin{document}
\selectlanguage{french}
\maketitle
\frontmatter
\clearemptydoublepage

%% Requis par les dispositions générales des travaux de Bachelor
\preamble
\authentification

%% Résumé / Version abbrégée
\begin{abstract}
% Francais
\lipsum[1]

%% L'asterisme est un signe typographique en forme d'étoile, utilisé pour marquer une pause dans un texte ou pour séparer des paragraphes. Il est souvent utilisé pour indiquer un changement de scène dans un récit. Bien qu'il se fasse rare dans la typographie moderne, c'est un symbole de choix pour séparer les différentes langues du résumé de thèse.
\asterism

% English
\lipsum[3]

\end{abstract}

%% Sommaire et tables
\clearemptydoublepage
{
    \tableofcontents
    \let\cleardoublepage\clearpage
    \listoffigures
    \let\cleardoublepage\clearpage
    \listoftables
    \let\cleardoublepage\clearpage
    \listoflistings
}

\printnomenclature
\clearemptydoublepage
\pagenumbering{arabic}

%% Contenu
\mainmatter
\chapter{Introduction}

L'introduction est une section requise dans un rapport technique. Introduisez votre travail, l'idée de départ et l'objectif attendu. Une personne qui découvrirait votre projet au travers de cette introduction devrait être capable d'en comprendre le cadre, l'idée générale et ses aboutissants.

\section{Contexte}

Cette section n'est pas obligatoire mais elle est souvent présente dans un rapport technique pour compléter l'introduction et définir le contexte du travail \cad le cadre formel dans lequel le travail est mené.

\section{Citations et bibliographie}

Vous le savez maintenant, citez vos sources est essentiel. La gestion d'une bibliographie n'est pas aisée mais l'utilisation de \texttt{biblatex} permet de facilement citer des article, des livres, des sites internet. Toutes les références citées seront automatiement regroupées en fin de document. Par exemple, citons un article d'Einstein \cite{einstein} ou le livre de Dirac \cite{dirac}.

Parfois il peut être utile d'utiliser un gestionnaire de bibliographie. La communauté scientifique affectionne particulièrement l'outil Zotero (\url{https://www.zotero.org/}) qui permet de gérer une bibliothèque numérique d'ouvrages et de références numériques. L'outil permet de générer une bibliographie compatible avec \LaTeX.

\section{Exemple d'équation}
L'une des principales force de \LaTeX est sa facilité d'écrire des équation. L'équation \ref{eq:1}, citée à titre d'exemple, représente la transformation de phase d'une lentille biconvexe. Pour rédiger une équation \LaTeX vous pouvez utiliser des outils en ligne tel que \url{https://www.latex4technics.com/}.

\begin{equation} \label{eq:1}
\begin{split}
L(x,y) &= \exp\left( - i\frac{{2\pi }}{\lambda }\left( {n\Delta \varphi (x,y) + \Delta {\varphi _0} - \Delta \varphi (x,y)} \right)\right)\\
 &= {\exp\left({i\frac{{2\pi }}{\lambda }\Delta {\varphi _0}}\right)}{\exp\left({ - i\frac{{2\pi }}{{\lambda f}}({x^2} + {y^2})}\right)}
\end{split}
\end{equation}

\section{Exemple de diagramme}

Les diagrammes de flux peuvent être réalisés en utilsant l'outil \url{draw.io}. Une exportation en \texttt{.xml} permet de garder les sources de la figure. Le rendu en \texttt{.pdf} sera réalisé à la volée à la compilation. L'intérêt est double : n'avoir qu'une source de vérité \cad pas d'image intermédiaire à stocker, et réduire la quantité d'information stockée.

Comme l'image est au format XML, les textes sont accessibles au correcteur orthographique et il vous est possible de facilement les modifier sans avoir à éditer l'image.

\figi{euclide.xml}{9cm}{Algorithme d'Euclide}

Notons qu'il est inutile d'insérer des images coloriées, là ou la couleur n'offre aucune valeur ajoutée ; évitez également les ombres. Enfin, préférez toujours des représentations vectorielles là ou c'est possible.

Voici un autre type de diagramme utile (figure \ref{sequence.xml}), celui d'une séquence UML.

\figi{sequence.xml}{8cm}{Diagramme de séquence}


\clearpage
\section{Exemple de figure}

Vous avez le choix de dessiner des graphiques manuellement en utilisant des outils de dessin vectoriel comme Inkscape ou Adobe Illustrator comme illustré à la figure \ref{plot.svg} ou alors, si vous disposez de Python ou Matlab, vous pouvez générer vos figures à la volée. Le code source \ref{python} permet de générer la figure \ref{bode.py}.


\fig{plot.svg}{Exemple de graphique plan}


\begin{listing}[h]
\inputminted[frame=single]{python}{assets/figures/bode.py}
\caption{Génération d'un diagramme de Bode \label{python}}
\end{listing}

\figi{bode.py}{12cm}{Diagramme de Bode}

Vous pouvez également utiliser TikZ pour créez vos propres schémas électriques et électroniques comme l'exemple \ref{circuit}.

\begin{figure}
\begin{center}
\begin{circuitikz}
    \draw
    (0,0) to [short, *-] (6,0)
    to [V, l_=$\mathrm{j}{\omega}_m \underline{\phi}^s_R$] (6,2)
    to [R, l_=$R_R$] (6,4)
    to [short, i_=$\underline{i}^s_R$] (5,4)
    (0,0) to [open, v^>=$\underline{u}^s_s$] (0,4)
    to [short, *- ,i=$\underline{i}^s_s$] (1,4)
    to [R, l=$R_s$] (3,4)
    to [L, l=$L_{\sigma}$] (5,4)
    to [short, i_=$\underline{i}^s_M$] (5,3)
    to [L, l_=$L_M$] (5,0);
\end{circuitikz}
\caption{Circuit électrique \label{circuit}}
\end{center}
\end{figure}

\section{Tableaux}

Pour les tableaux, restez simple et minimaliste, n'ajoutez des séparateurs que là ou c'est nécessaire pour améliorer la lisibilité. Une liste de quelques cantons suisse est donné à titre d'exemple dans la table \ref{cantons}.

\begin{table}[h]
\begin{center}
\caption{Liste des cantons \label{cantons}}
\begin{tabular}{|c|l|r|}
Abréviation & Nom du canton & Depuis \\ \hline
ZH & Zürich & \ordinalnum{1} mai 1351 \\
BE & Berne & 6 mars 1353 \\
FR & Fribourg & 22 décembre 1481 \\
VD & Vaud & 19 février 1815 \\
VS & Valais & 4 août 1815 \\
NE & Neuchâtel & 19 mai 1815 \\
GE & Genève & 19 mai 1815
\end{tabular}
\end{center}
\end{table}

\section{Index}
\LaTeX~ permet d'indexer les mots \index{mots} importants. Il suffit de placer les termes importants d'un paragraphe dans la commande \texttt{\textbackslash index\{terme\}} et ils apparaîtront automatiquement à la fin de ce rapport dans l'index du document.

\index{Napoléon}

Imaginons que dans cette section nous parlions du cheval blanc \index{cheval blanc} de Napoléon. Il se pourrait que le lecteur recherche ce passage dans la version imprimée du rapport. Avec l'index, rien de plus facile. Allez jeter un oeil à la page \pageref{index}.

\section{Notes de bas de page}

Parfois, il est plus élégant d'annoter une définition en utilisant une note de bas de page \footnote{La note en bas de page (ou note de bas de page) est une forme littéraire, consistant en une ou plusieurs lignes ne figurant pas dans le texte.}. Alternativement il est possible d'anoter le paragraphe courant avec une note marginale. \maraja{Note marginale}

\section{Glossaire et acronymes}

La \Gls{heig-vd} membre de la \Gls{hes-so} propose ce modèle de document. Le format \Gls{latex} est particulièrement adapté pour les documents qui contiennent des expressions \gls{maths}. Pour plus de détail sur l'utilisation d'un glossaire, se référer à \url{https://www.overleaf.com/learn/latex/Glossaries}.

\chapter{Discussion}
\section{Résultats}
\lipsum[10]

\section{Conclusion}
\lipsum[11]

\vfil
\hspace{8cm}John Doe\par
\hspace{8cm}\begin{minipage}{5cm}
% Place pour signature numérique
\end{minipage}
\clearpage

\appendix
\appendixpage
\addappheadtotoc

\chapter{Première annexe}
\section{Sous section}
\lipsum[1]

\let\cleardoublepage\clearpage
\backmatter

\printglossary
\printbibliography
\label{index}
\printindex

\end{document}
