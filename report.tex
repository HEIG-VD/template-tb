\documentclass{heig-tb}


\usepackage[europeanresistors,americaninductors]{circuitikz}

\makenomenclature
\makeglossaries
\makeindex

\addbibresource{bibliography.bib}

%% Défini le style pour les extraits de code
\lstset{%
language=C, % Vous pouvez choisir le langage de votre choix
breaklines=true,
breakatwhitespace=false,
basicstyle=\footnotesize\sffamily}

%% Quelques entrées de nomenclature
\usepackage{etoolbox}
\renewcommand\nomgroup[1]{%
  \item[\bfseries
  \ifstrequal{#1}{A}{Constantes physiques}{%
  \ifstrequal{#1}{B}{Groupes}{%
  \ifstrequal{#1}{C}{Autres Symboles}{}}}%
]}

\newcommand{\nomunit}[1]{%
\renewcommand{\nomentryend}{\hspace*{\fill}#1}}

\nomenclature[A, 02]{\(c\)}{\href{https://physics.nist.gov/cgi-bin/cuu/Value?c}
{Vitesse de la lumière dans le vide}
\nomunit{\SI{299792458}{\meter\per\second}}}

\nomenclature[A, 03]{\(h\)}{\href{https://physics.nist.gov/cgi-bin/cuu/Value?h}
{Constante de Planck}
\nomunit{\SI[group-digits=false]{6.62607015e-34}{\joule\per\hertz}}}

\nomenclature[A, 01]{\(G\)}{\href{https://physics.nist.gov/cgi-bin/cuu/Value?bg}
{Constante de gravitation universelle}
\nomunit{\SI[group-digits=false]{6.67430e-11}{\meter\cubed\per\kilogram\per\second\squared}}}

\nomenclature[B, 03]{\(\mathbb{R}\)}{Nombres réels}
\nomenclature[B, 02]{\(\mathbb{C}\)}{Nombres complexes}
\nomenclature[B, 01]{\(\mathbb{H}\)}{Quaternions}

\nomenclature[C]{\(V\)}{Volume constant}
\nomenclature[C]{\(\rho\)}{Indice de frottement sec}

%% Quelques acronymes
\newacronym{gcd}{GCD}{Plus grand diviseur commun}
\newacronym{lcm}{LCM}{Plus petit multiple commun}

%% Quelques entrées de glossaire
\newglossaryentry{heig-vd}{
    name=HEIG-VD,
    description={Haute École d'Ingénierie et de Gestion du canton de Vaud}}
\newglossaryentry{hes-so}{
    name=HES-SO,
    description={Haute École Supérieure de Suisse Occidentale}}

\date{\today} % Mettre à jour avec la date de rendu du travail

% Auteur du document (étudiant-e) en projet de Bachelor
\author{Yves Chevallier}

IIDE, REDS, IAI, IGT, MEI, IICT, INSIT, IESE, COMATEC

% Acronyme de l'institut
\institute{IAI}
\endyear{2021}
\department{TIN}
\faculty{Génie électrique}
\orientation{Électronique embarquée et mécatronique}
\thesis{7212}

% Choisir l'option idoine (utilisé pour l'accord féminin)
\genre{male}
%\genre{female}

% Choisir l'option idoine
\field{Ingénierie}
%\field{Economie d'entreprise}

\title{Titre de projet de Bachelor}
\subtitle{Travail de Bachelor}

% Nom du professeur responsable
\teacher {Prof. Y. Chevallier (HEIG-VD)}


%% Début du document
\begin{document}
\selectlanguage{french}
\maketitle
\frontmatter
\clearemptydoublepage

%% Requis par les dispositions générales
\preamble
\authentification

%% Résumé / Abstract
\begin{abstract}
% Francais
\lipsum[1]

%% L'asterisme est un signe typographique en forme d'étoile, utilisé pour marquer une pause dans un texte ou pour séparer des paragraphes. Il est souvent utilisé pour indiquer un changement de scène dans un récit. Bien qu'il se fasse rare dans la typographie moderne, c'est un symbole de choix pour séparer les différentes langues du résumé de thèse.
\asterism

% English
\lipsum[3]

\end{abstract}

%% Table des matières
\clearemptydoublepage

\tableofcontents

\listoffigures
\listoftables

\printnomenclature

\clearemptydoublepage
\pagenumbering{arabic}

%% Contenu - Content
\mainmatter
\chapter{Introduction}
\lipsum[3-4]

\section{Contexte}
\lipsum[5]

\section{But du projet}
\lipsum[7]

\begin{enumerate}
\item \lipsum[6][1]
\item \lipsum[6][2]
\item \lipsum[6][3]
\end{enumerate}

\chapter{Analyse}
\section{Introduction}

\lipsum[8]

\section{Citations et bibliographie}

L'utilisation de \texttt{biblatex} permet de regrouper les références citées en fin de document. Citons par exemple l'article d'Einstein
\cite{einstein} et le livre de Dirac \cite{dirac}. Tous deux sont liés à la physique. Ensuite, mentionnons le \textit{The \LaTeX\ Companion}
 \cite{latexcompanion}, et le site internet de Donald Knuth's \cite{knuthwebsite}. N'oublions pas le
\textit{The Comprehensive Tex Archive Network} (CTAN)
\cite{ctan}.

\section{Exemple d'équation}
L'équation \ref{eq:1} représente la transformation de phase d'une lentille biconvexe.

\begin{equation} \label{eq:1}
\begin{split}
L(x,y) &= \exp\left( - i\frac{{2\pi }}{\lambda }\left( {n\Delta \varphi (x,y) + \Delta {\varphi _0} - \Delta \varphi (x,y)} \right)\right)\\
 &= \exp\left({{i\frac{{2\pi }}{\lambda }\Delta {\varphi _0}}}\right)\exp\left({{ - i\frac{{2\pi }}{\lambda }(n - 1)\Delta \varphi (x,y)}}\right)\\
 &= {\exp\left({i\frac{{2\pi }}{\lambda }\Delta {\varphi _0}}\right)}{\exp\left({ - i\frac{{2\pi }}{{\lambda f}}({x^2} + {y^2})}\right)}
\end{split}
\end{equation}

Pour rédiger une équation \LaTeX vous pouvez utiliser des outils en ligne tel que \url{https://www.latex4technics.com/}.

\clearpage
\section{Exemple de diagramme de flux}

Les diagrammes de flux peuvent être réalisés en utilsant l'outil \url{draw.io}. Une exportation en \texttt{.xml} permet de garder les sources de la figure. Le rendu en \texttt{.pdf} sera réalisé à la volée à la compilation.

\figi{euclide.xml}{10cm}{Algorithme d'Euclide}

Inutile d'insérer des images coloriées là ou la couleur n'offre aucune valeur ajoutée.

Préférez toujours des représentations vectorielles là ou c'est possible.

\clearpage
\section{Exemple de figure}

Vous avez le choix de dessiner des graphiques manuellement en utilisant des outils de dessins comme Inkscape ou Adobe Illustrator comme illustré à la figure \ref{plot}.

\fig{plot.svg}{Exemple de graphique plan}

Alternativement, si vous disposez de Python ou Matlab, vous pouvez générer vos figures à la volée. À titre d'exemple, le code suivant permet de générer la figure \ref{bode}.

\lstinputlisting[language=Python]{assets/bode.py.pdf}

\fig{bode}{Diagramme de Bode}

Vous pouvez également utiliser TikZ pour créez vos propres schémas électriques et électroniques comme l'exemple \ref{circuit}

\begin{figure}
\begin{center}
\begin{circuitikz}[american voltages]
    \draw
        (0,0) to [short, *-] (6,0)
        to [V, l_=$\mathrm{j}{\omega}_m \underline{\psi}^s_R$] (6,2)
        to [R, l_=$R_R$] (6,4)
        to [short, i_=$\underline{i}^s_R$] (5,4)
        (0,0) to [open, v^>=$\underline{u}^s_s$] (0,4)
        to [short, *- ,i=$\underline{i}^s_s$] (1,4)
        to [R, l=$R_s$] (3,4)
        to [L, l=$L_{\sigma}$] (5,4)
        to [short, i_=$\underline{i}^s_M$] (5,3)
        to [L, l_=$L_M$] (5,0);
        \end{circuitikz}
\caption{Circuit électrique \label{circuit}}
\end{center}
\end{figure}

\section{Tableaux}

\begin{table}
\begin{center}
\caption{Liste des cantons \label{cantons}}
\begin{tabular}{|c|l|r|}
Abréviation & Nom du canton & Depuis \\ \hline
ZH & Zürich & \ordinalnum{1} mai 1351 \\
BE & Berne & 6 mars 1353 \\
FR & Fribourg & 22 décembre 1481 \\
VD & Vaud & 19 février 1815 \\
VS & Valais & 4 août 1815 \\
NE & Neuchâtel & 19 mai 1815 \\
GE & Genève & 19 mai 1815
\end{tabular}
\end{center}
\end{table}

\section{Index}
\LaTeX dispose d'un index automatique des termes \index{termes} que vous souhaitez. Il suffit de placer le terme dans la commande \texttt{\textbackslash index\{terme\}}. Il apparaîtra automatiquement à la fin de ce rapport dans l'index du document.

Imaginons que dans cette section nous parlions du cheval blanc \index{cheval blanc} de Napoléon \index{Napoléon}. Il se pourrait que le lecteur recherche ce passage dans la version imprimée du rapport. Avec l'index, rien de plus facile. Allez jeter un oeil à la page \pageref{index}.

\section{Notes de bas de page}

Parfois, il est plus élégant d'annoter une définition en utilisant une note de bas de page \footnotemark.

Alternativement il est possible d'anoter le paragraphe courant avec une note marginale. \maraja{Note marginale}

\footnotetext{La note en bas de page (ou note de bas de page) est une forme littéraire, consistant en une ou plusieurs lignes ne figurant pas dans le texte.}

\section{Glossaire et acronymes}

Le format \Gls{latex} est particulièrement adapté pour les documents qui contiennent des expressions \gls{maths}. Les \Glspl{formula} sont affichées élégament. Pour plus de détail sur l'utilisation d'un glossaire, se référer à \url{https://www.overleaf.com/learn/latex/Glossaries}.

\chapter{Discussion}
\section{Résultats}

\lipsum[10]

\section{Conclusion}

\lipsum[11]

\hspace{7cm}John Doe\par
\hspace{7cm}\begin{minipage}{5cm}

\end{minipage}

\clearpage

\appendix
\appendixpage
\addappheadtotoc

\chapter{\lipsum[3][3]}

\section{\lipsum[3][1]}

\lipsum[1-2]


\backmatter


\printglossary
\printbibliography
\label{index}
\printindex

\end{document}
