\chapter{C'est quoi une annexe ?}

Précisons-le tout de toute, les annexes ne possèdent pas de contenu \underline{normatif} mais \underline{descriptif}. Tout contenu annexé ne doit pas être indispensable à la compréhension du travail réalisé, elles sont ajoutées en complément du rapport.

Les annexes contiennent généralement :

\begin{itemize}
    \item les dessins mécaniques (mises en plan);
    \item les schémas électriques détaillés;
    \item des photographies du projet;
    \item des scripts et des extraits de code source;
    \item des documents techniques \pex \emph{datasheet};
    \item des développements mathématiques.
\end{itemize}

Les annexes sont utiles pour reproduire le travail réalisé, pour le comprendre en profondeur.

Il est courant d'avoir plusieurs annexes, chacune étant numérotée et identifiée par une lettre.

Évitez également d'être trop généreux dans vos annexes, elles ne doivent pas être un fourre-tout et il peut être mal vu d'avoir un rapport de 50 pages avec 200 pages d'annexes. Préférez des ressources en lignes, mentionnez vos sources, vos dépôts de code (GitHub, GitLab, Bitbucket, etc.). Profitez de donner les hash de vos commits pour que le lecteur puisse retrouver exactement la version de votre code source utilisée au moment de la rédaction du rapport.