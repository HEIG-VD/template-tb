\chapter{Justification du modèle}

Il est nécessaire, si ce n'est indispensable de justifier le choix de ce modèle de document, notament en terme de style. En art graphique, en inforgraphie, les goûts et les couleurs ne se discutent pas néanmoins on peut sans risque d'être biaisé par l'opinion d'affirmer qu'il existe des concensus et des règles typographiques séculaires largement adoptées par la communauté scientifique et académique.

Tout d'abord le choix de \LaTeX{} qui reste dominant pour l'écriture d'articles scientifiques. De préstigieuses revues scientifiques comme Nature, Science, IEEE Journals ou Elsevier acceptent le plus volontiers des articles rédigés en \LaTeX{}. Les raisons sont nombreuses mais notament car \LaTeX{} offre une meilleure gestion des équations complexes, une mise en page automatique conformes aux exigences des revues, une gestion efficace des références bibliographiques et une qualité typographique supérieure à celle d'autres outils de traitement de texte, notamment par le support des voeuves et des orphelines, des césures de mots ou des ligatures. Les ajustements précis de l'espacement entre les lettres et les mots rend l'écrit plus harmonieux.

Cette popularité et cet attachement aux conventions d'édition héritées de de l'imprimerie traditionnelle reste un gage de qualité et de sérieux pour les lecteurs et les pairs.

\section{Police de caractère}

Il n'existe guère de règle, chaque université, chaque journal et chaque éditeur se distingue par le choix d'une police de caractère particulière. Nous avons fait le choix ici de rester fidèle à la tradition en adoptant la famille Computer Modern qui est la police par défaut de \LaTeX{}. Elle a été conçue par Donald Knuth pour son système de composition de texte \TeX{} et il s'agit de la police la plus couramment utilisée dans les articles scientifiques. Elle est appréciée pour sa lisibilité et son esthétique.

Les conventions typographiques recommandent néanmoins de ne pas surcharger ni le nombre ni la variété des polices de caractères. Il souvent recommandé d'utiliser une seule famille de police pour l'ensemble du document, et le plus important d'éviter l'usage de la couleur pour les titres et les textes. Les titres doivent être mis en valeur par leur taille et leur graisse, et les textes doivent être noirs pour une lecture optimale.

\section{En-tête et pieds de page}

Le choix du style d'en-tête a été simple, il est la configuration par défaut de \LaTeX{} pour les documents de type \texttt{book}.

Contrairement aux allégations un peu hardies de certains, on ne met pas le logo de l'école ou de l'entreprise sur chaque page d'un rapport, on ne surcharge non plus l'en-tête ou le pied de page d'informations redondantes tel que le nom ou le titre de l'ouvrage. Un document d'un seul tenant (livre, manuscrit, PDF) a peu de chance d'être désolidarisé et de perdre sa page de couverture. Il est par conséquent inutile d'y répéter de l'informations qui s'y trouvent déjà.

\section{Découpe en trois parties}

Traditionnellement, et ce depuis la rennaissance (XVe) un ouvrage est découpé en trois parties nommées en anglais \emph{frontmatter} (préliminaires) qui précède le coeur du texte. Il contient les pages liminaires qui servent d'instroduction au conctenu principal. On y retrouve les tables des matières et des figures, les remerciements, les dédicaces, les préfaces, les résumés, etc. La numérotation se fait généralement en chiffres romains minuscules (i, ii, iii, iv, etc.). La pagination débute à partir de cette section, mais les pages blanches (versos) ne sont généralement pas numérotées. Les titres de section ne sont généralement pas numérotés.

Le \emph{mainmatter} correspond au coeur du texte, où le contenu principal est développé. La numérotation des pages est en chiffres indo-arabes (1, 2, 3, ...). Les chapitres commencent généralement sur une page impaire (à droite) et les en-têtes contiennent souvent le titre du chapitre sur les pages paires et le titre de la section sur les pages impaires.

Le \emph{backmatter} est la section finale du document, qui contient les annexes, la bibliographie, les index, les glossaires, etc. La numérotation des pages continue avec la partie précédante, le ton est généralement plus technique et informatif. On y retrouve généralement les notes de fin, la bibliographie, l'index, le glossaire et les annexes.

\section{Références (figures et tables)}

Les figures et les tables doivent être numérotées et référencées dans le texte.
La légende d'une figure est généralement placée en dessous de la figure, et la légende d'une table est généralement placée au-dessus de la table. Cette convention de longue date est retrouvée dans de nombreux manuels de styles comme APA, Chivago ou MALA.

\section{Mise en page}
