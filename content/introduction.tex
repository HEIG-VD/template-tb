\chapter{Introduction}

L'introduction est une section \emph{obligatoire} dans un rapport technique. Elle contient généralement au moins quatre paragraphes répondant aux questions suivantes: quel est le problème à résoudre ? Pourquoi y a-t-il un problème ? Quelle est la solution envisagée ? Pourquoi est-ce que la solution envisagée est meilleure que l'état de l'art ?

En bref, on vous demande d'introduire votre travail, l'idée de départ et les objectifs attendus. Le lecteur qui découvrirait votre projet au travers de cette introduction devrait ainsi être capable d'en comprendre le cadre, l'idée générale et les aboutissants du projet.

\section{Contexte}

Cette section \emph{n'est pas obligatoire}, mais elle est souvent présente en préambule de l'introduction afin de préciser le contexte du travail \cad et le cadre formel dans lequel le travail est mené. Dans bien des cas, le projet est réalisé dans un contexte industriel, vis-à-vis d'une problématique un peu en marge avec l'intitulé du projet. Le contexte permet de clarifier les enjeux et les contraintes du projet.

\section{Problématique}

La problématique est le cœur de l'introduction. Elle doit être clairement formulée et doit permettre de comprendre le problème que vous vous apprêtez à résoudre. Il est important de bien définir le problème pour que le lecteur puisse comprendre les enjeux du projet et les raisons pour lesquelles vous avez choisi de travailler sur ce sujet et pourquoi vous y amenez une valeur ajoutée.

\section{Objectifs}

On attend généralement une liste d'objectifs clairs et précis. Cette liste doit être cohérente avec le contexte et les enjeux du projet. Les objectifs doivent être mesurables et atteignables. Ils doivent être formulés de manière à ce qu'ils puissent être validés à la fin du projet.

\section{Méthodologie}

Vous pouvez ajouter une section méthodologie expliquant votre démarche pour atteindre les objectifs. Cette section peut être utile pour expliquer les choix que vous avez faits, les outils que vous avez utilisés, les raisons pour lesquelles vous avez choisi une approche plutôt qu'une autre.
